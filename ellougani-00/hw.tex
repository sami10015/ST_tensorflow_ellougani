\documentclass[a4paper]{article}
\usepackage[letterpaper, margin=1in]{geometry} % page format
\usepackage{listings} % this package is for including code
\usepackage{graphicx} % this package is for including figures
\usepackage{amsmath}  % this package is for math and matrices
\usepackage{amsfonts} % this package is for math fonts
\usepackage{tikz} % for drawings
\usepackage{hyperref} % for urls

\title{Homework 0}
\author{Sami Ellougani}
\date{9/09/17}

\begin{document}
\lstset{language=Python}

\maketitle

\section{Solution to Problem 1}
The problem was to find the value for x that maximizes g(x) where g(x) = $-3x^2 + 24x - 30$. First, you need to 
take the derivate and equal it to zero to solve for critical points. This is $24-6x = 0$. Now, we subtract 24 and then divide by negative
6 to prove that the value that maximizes x is equal to 4.

\section{Solution to Problem 2}
The problem was to find the value for x that maximizes g(x) where g(x) = $-3x^2 + 24x - 30$. First, you need to 
take the derivate and equal it to zero to solve for critical points. This is $24-6x = 0$. Now, we subtract 24 and then divide by negative
6 to prove that the value that maximizes x is equal to 4.

\section{Solution to Problem 3}
\subsection{} 
You are unable to multiply the two matrices because the number of columns in matrix A is not equal to the number of rows in matrix B.
\subsection{}
The first step in the is problem is to find the transpose of matrix A. The transpose of matrix A is:
\[
A^T=
  \begin{bmatrix}
    1 & 2  \\
    4 & -1 \\
   -3 & 3
  \end{bmatrix}
\]
Now that the colums match the number of rows in matrix B, we can now multiply them togethering resulting:
\[
  \begin{bmatrix}
    -2 & -2 & 13 \\
    -8 & 1 & 16 \\
    6 & -3 & -3
  \end{bmatrix}
\]
To calculate the rank, we simply transform the matrix to its row echolon form and count the number of non-zero rows. We get an answer of 2 for our rank.

\break
\subsection{Code}
Below, I show two code snippets of me multiplying $A^T$ and B in Numpy along with Tensorflow.

\begin{lstlisting}[frame=single]
#By Sami Ellougani
import numpy as np
import tensorflow as tf

#Multiplying matrix's with numpy
a = np.array([[1, 2], [4, -1], [-3, 3]])
b = np.array([[-2, 0, 5], [0, -1, 4]])
print(np.dot(a,b))

#Multiplying matrix's with tensorflow
a = tf.constant(a)
b = tf.constant(b)

with tf.Session() as sess:
	print(tf.matmul(a,b).eval())
\end{lstlisting}
Here are my reults from the command line:

\begin{lstlisting}
> python matrixMult.py
[[-2 -2 13]
[-8 1 16]
[6 -3 -3]]

[[-2 -2 13]
[-8 1 16]
[6 -3 -3]]
\end{lstlisting}

\section{Solution to Problem 4}


\end{document}
